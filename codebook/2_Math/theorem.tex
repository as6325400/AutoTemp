\begin{itemize}
\item Cramer 法則
$$
\begin{aligned}ax+by=e\\cx+dy=f\end{aligned}
\Rightarrow
\begin{aligned}x=\dfrac{ed-bf}{ad-bc}\\y=\dfrac{af-ec}{ad-bc}\end{aligned}
$$

\item Vandermonde 恆等式
$$
C(n+m,k)=\sum_{i=0}^{k} C(n,i)\,C(m,k-i)
$$

\item Kirchhoff 定理

設 $L$ 為圖 $G$ 的 $n\times n$ Laplacian 矩陣,其中 $L_{ii}=d(i)$,$L_{ij}=-c$,$c$ 為邊 $(i,j)$ 的條數。
\begin{itemize}
    \item 無向圖的生成樹數為 $\lvert \det(\tilde{L}_{11}) \rvert$。
    \item 以 $r$ 為根的有向生成樹數為 $\lvert \det(\tilde{L}_{rr}) \rvert$。
\end{itemize}

\item Tutte 矩陣

令 $D$ 為 $n\times n$ 矩陣,當 $i<j$ 且 $(i,j)\in E$ 時令 $d_{ij}=x_{ij}$($x_{ij}$ 由均勻隨機選取),否則令 $d_{ij}=-d_{ji}$。則 $\dfrac{\mathrm{rank}(D)}{2}$ 為圖 $G$ 的最大匹配數。

\item Cayley 公式

\begin{itemize}
  \item 對於帶標號頂點、度數序列為 $d_1,d_2,\ldots,d_n$ 的情形,生成樹數為 $\dfrac{(n-2)!}{(d_1-1)!(d_2-1)!\cdots(d_n-1)!}$。
  \item 設 $T_{n,k}$ 為在 $n$ 頂點、$k$ 個連通分量的帶標號森林數,且頂點 $1,2,\ldots,k$ 分屬不同分量,則 $T_{n,k}=k\,n^{\,n-k-1}$。
\end{itemize}

\item Erdős–Gallai 定理

非負整數序列 $d_1\ge\cdots\ge d_n$ 可表示為 $n$ 頂點簡單圖的度數序列,當且僅當 $d_1+\cdots+d_n$ 為偶數,且對每個 $1\le k\le n$ 皆有
\[
\textstyle\sum_{i=1}^k d_i \le k(k-1)+\sum_{i=k+1}^n \min(d_i,k)\,.
\]

\item Gale–Ryser 定理

非負整數序列 $a_1\ge\cdots\ge a_n$ 與 $b_1,\ldots,b_n$ 構成雙圖度數序列(bigraphic)的充要條件為
\[
\textstyle\sum_{i=1}^n a_i=\sum_{i=1}^n b_i
\quad\text{且}\quad
\sum_{i=1}^k a_i\le \sum_{i=1}^n\min(b_i,k)\ \text{對所有 }1\le k\le n.
\]

\item Fulkerson–Chen–Anstee 定理

一序列 $(a_1,b_1),\ldots,(a_n,b_n)$(非負整數對,且 $a_1\ge\cdots\ge a_n$)為有向圖可實現之入出度序列,當且僅當
\[
\textstyle\sum_{i=1}^n a_i=\sum_{i=1}^n b_i
\quad\text{且}\quad
\sum_{i=1}^k a_i\le \sum_{i=1}^k\min(b_i,k-1)+\sum_{i=k+1}^n\min(b_i,k)\ \text{對所有 }1\le k\le n.
\]

\item Pick 定理

對所有頂點為整數點的單純多邊形,有
\[
A=\#\{\text{內部格點}\}+\frac{\#\{\text{邊界格點}\}}{2}-1.
\]

\item 莫比烏斯反演公式

\begin{itemize}
  \item $f(n)=\sum_{d\mid n}g(d)\ \Leftrightarrow\ g(n)=\sum_{d\mid n}\mu(d)\,f\!\left(\dfrac{n}{d}\right)$
  \item $f(n)=\sum_{n\mid d}g(d)\ \Leftrightarrow\ g(n)=\sum_{n\mid d}\mu\!\left(\dfrac{d}{n}\right)f(d)$
\end{itemize}

\item 球冠(Spherical cap)

\begin{itemize}
  \item 球面被平面切下的一部分。
  \item $r$: 球半徑,$a$: 球冠底面半徑,$h$: 球冠高度,$\theta=\arcsin(a/r)$。
  \item 體積 $=\dfrac{\pi h^2(3r-h)}{3}=\dfrac{\pi h(3a^2+h^2)}{6}=\dfrac{\pi r^3(2+\cos\theta)(1-\cos\theta)^2}{3}$。
  \item 表面積 $=2\pi rh=\pi(a^2+h^2)=2\pi r^2(1-\cos\theta)$。
\end{itemize}

\item 拉格朗日乘數法

\begin{itemize}
  \item 在 $k$ 個約束 $g_i(x_1,\ldots,x_n)=0$ 下優化 $f(x_1,\ldots,x_n)$。
  \item 拉格朗日函數 $\mathcal{L}(x_1,\ldots,x_n,\lambda_1,\ldots,\lambda_k)=f(x_1,\ldots,x_n)-\sum_{i=1}^{k}\lambda_i\,g_i(x_1,\ldots,x_n)$。
  \item 對應原始受限問題的解是拉格朗日函數的一個鞍點。
\end{itemize}

\item 兩條錯線的最近點

\begin{itemize}
\item $\text{Line 1}: \boldsymbol{v}_1=\boldsymbol{p}_1+t_1\boldsymbol{d}_1$
\item $\text{Line 2}: \boldsymbol{v}_2=\boldsymbol{p}_2+t_2\boldsymbol{d}_2$
\item $\boldsymbol{n}=\boldsymbol{d}_1\times \boldsymbol{d}_2$
\item $\boldsymbol{n}_1=\boldsymbol{d}_1\times \boldsymbol{n}$
\item $\boldsymbol{n}_2=\boldsymbol{d}_2\times \boldsymbol{n}$
\item $\boldsymbol{c}_1=\boldsymbol{p}_1+\dfrac{(\boldsymbol{p}_2-\boldsymbol{p}_1)\cdot\boldsymbol{n}_2}{\boldsymbol{d}_1\cdot\boldsymbol{n}_2}\,\boldsymbol{d}_1$
\item $\boldsymbol{c}_2=\boldsymbol{p}_2+\dfrac{(\boldsymbol{p}_1-\boldsymbol{p}_2)\cdot\boldsymbol{n}_1}{\boldsymbol{d}_2\cdot\boldsymbol{n}_1}\,\boldsymbol{d}_2$
\end{itemize}

\item 導數/積分

分部積分:
\(\int_a^b f(x)g(x)\,dx=[F(x)g(x)]_a^b-\int_a^b F(x)g'(x)\,dx\)

{
  \setlength{\tabcolsep}{1pt}
  \setlength{\columnsep}{0pt}

  \noindent
  \begin{tabular}{|*{20}{>{$\displaystyle}c<{$}|}}
    \frac{d}{dx}\sin^{-1} x = \frac{1}{\sqrt{1-x^2}}
    &
    \frac{d}{dx}\cos^{-1} x = -\frac{1}{\sqrt{1-x^2}}
    &
    \frac{d}{dx}\tan^{-1} x = \frac{1}{1+x^2}
    \\
    \frac{d}{dx}\tan x = 1+\tan^2 x
    &
    \int \tan(ax)\,dx = -\frac{\ln|\cos(ax)|}{a}
    &
    \\
    \int e^{-x^2}\,dx = \frac{\sqrt \pi}{2}\,\operatorname{erf}(x)
    &
    \int x e^{ax}\,dx = \frac{e^{ax}}{a^2}(ax-1)
    \\
  \end{tabular}
  
  \(
    \displaystyle
    \int \sqrt{a^2 + x^2}\,dx = \frac{1}{2}\!\left(x\sqrt{a^2+x^2} + a^2 \operatorname{asinh}(x/a) \right)
  \)
}

\item 球座標(Spherical Coordinate)

$$
(x,y,z)=(r\sin\theta\cos\phi,\ r\sin\theta\sin\phi,\ r\cos\theta)
$$

$$
(r,\theta,\phi)=\bigl(\sqrt{x^2+y^2+z^2},\ \arccos(z/\sqrt{x^2+y^2+z^2}),\ \operatorname{atan2}(y,x)\bigr)
$$

\item 旋轉矩陣

$$
M(\theta)=
\begin{bmatrix}
\cos\theta & -\sin\theta\\
\sin\theta & \cos\theta
\end{bmatrix},
\quad
R_x(\theta_x)=
\begin{bmatrix}
1 & 0 & 0\\
0 & \cos\theta_x & -\sin\theta_x \\
0 & \sin\theta_x & \cos\theta_x
\end{bmatrix}
$$

\end{itemize}
