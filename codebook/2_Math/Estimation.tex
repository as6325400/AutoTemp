{
  \setlength{\tabcolsep}{1pt}
  \setlength{\columnsep}{0pt}

  % 整數分割數 p(n)
  \noindent\textbf{整數分割數 $p(n)$:}把 $n$ 分成正整數和的不同方式數量。
  \begin{tabular}{@{}c|*{20}{c@{\ }}@{}}
    $n$    & 2 & 3 & 4 & 5 & 6  & 7  & 8  & 9  & 20  & 30   & 40  & 50  & 100 \\
    \hline
    $p(n)$ & 2 & 3 & 5 & 7 & 11 & 15 & 22 & 30 & 627 & 5604 & 4e4 & 2e5 & 2e8 \\
  \end{tabular}

  \vspace{0.4em}

  % 最大因數數量函數 d(i)
  \noindent\textbf{最大因數數量 $\max_{i\le n} d(i)$:}小於等於 $n$ 的整數中,因數最多者的因數個數。
  \begin{tabular}{@{}c|*{20}{c@{\ }}@{}}
    $n$
    & 100 & 1e3 & 1e6 & 1e9  & 1e12 & 1e15  & 1e18 \\
    \hline
    $d(i)$
    & 12  & 32  & 240 & 1344 & 6720 & 26880 & 103680 \\
  \end{tabular}

  \vspace{0.4em}

  % 中央二項係數
  \noindent\textbf{中央二項式 $\binom{2n}{n}$:}常用於卡塔蘭數與組合計數。
  \begin{tabular}{c|*{20}c}
    $n$             & 1 & 2 & 3  & 4  & 5   & 6   & 7    & 8     & 9
                    & 10     & 11  & 12  & 13  & 14  & 15 \\
                    \hline
    $\binom{2n}{n}$ & 2 & 6 & 20 & 70 & 252 & 924 & 3432 & 12870 & 48620
                    & 184756 & 7e5 & 2e6 & 1e7 & 4e7 & 1.5e8 \\
  \end{tabular}

  \vspace{0.4em}

  % 貝爾數
  \noindent\textbf{貝爾數 $B_n$:}將 $n$ 個元素分成任意多個非空集合的劃分數。
  \begin{tabular}{c|*{20}c}
    $n$             & 2 & 3  & 4  & 5   & 6   & 7    & 8     & 9 & 10     & 11  & 12  & 13  \\
                    \hline
    $B_n$           & 2 & 5 & 15 & 52 & 203 & 877 & 4140 & 21147 & 115975 & 7e5 & 4e6 & 3e7 \\
  \end{tabular}
}
