\begin{itemize}
\item 白努力數(Bernoulli numbers)

$B_0=1,\ B_1^{\pm}=\pm\frac{1}{2},\ B_2=\frac{1}{6},\ B_3=0$

滿足關係式 $\displaystyle\sum_{j=0}^m\binom{m+1}{j}B_j=0$。  
其指數生成函數(EGF)為
\[
B(x)=\frac{x}{e^x-1}=\sum_{n=0}^{\infty}B_n\frac{x^n}{n!}.
\]
且
\[
S_m(n)=\sum_{k=1}^{n}k^m=\frac{1}{m+1}\sum_{k=0}^{m}\binom{m+1}{k}B_k^{+}\,n^{m+1-k}.
\]

\item 第二類史特林數(Stirling numbers of the second kind)

表示將 $n$ 個不同元素分成恰好 $k$ 個非空集合的劃分數。  
遞迴式:
\[
S(n,k)=S(n-1,k-1)+k\,S(n-1,k), \quad S(n,1)=S(n,n)=1.
\]
顯式公式:
\[
S(n,k)=\frac{1}{k!}\sum_{i=0}^{k}(-1)^{k-i}\binom{k}{i}i^n.
\]
轉換關係:
\[
x^n=\sum_{i=0}^{n}S(n,i)\,(x)_i.
\]

\item 五角數定理(Pentagonal number theorem)

\[
\prod_{n=1}^{\infty}(1-x^n)
=1+\sum_{k=1}^{\infty}(-1)^k\!\left(x^{k(3k+1)/2}+x^{k(3k-1)/2}\right).
\]
此恆等式由歐拉提出,與分割函數的生成函數密切相關。

\item 卡塔蘭數(Catalan numbers)

\[
C^{(k)}_n=\frac{1}{(k-1)n+1}\binom{kn}{n},
\qquad
C^{(k)}(x)=1+x\,[C^{(k)}(x)]^k.
\]
其中 $C^{(2)}_n$ 為經典卡塔蘭數,常出現在二元樹、括號配對、Dyck path 等組合結構中。

\item 歐拉數(Eulerian numbers)

表示排列 $\pi\in S_n$ 中,恰有 $k$ 個元素比前一個元素大的排列數。  
滿足遞迴式:
\[
E(n,k)=(n-k)E(n-1,k-1)+(k+1)E(n-1,k),
\quad
E(n,0)=E(n,n-1)=1.
\]
顯式公式:
\[
E(n,k)=\sum_{j=0}^{k}(-1)^j\binom{n+1}{j}(k+1-j)^n.
\]

\end{itemize}
